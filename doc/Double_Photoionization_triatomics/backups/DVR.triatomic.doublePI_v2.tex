\documentclass[%
%prl%  
pra%
%rmp%
% ,preprint%
,twocolumn%
%,secnumarabic%
%,tightenlines%
%,byrevtex 
,amssymb, nobibnotes, aps,
longbibliography
]{revtex4-1}
%
\usepackage{amsmath}
\usepackage{euscript}
\usepackage{graphicx}
\usepackage[usenames]{color}
\usepackage{bbold}
%
\begin{document}
\title{
Double photoionization of a triatomic molecule  treating two active electrons: DVR and orbital basis
}
\author{CWM begun June 2020 }
%
\begin{abstract}
These notes will lay out all aspects of double photoionization calculations on water
\begin{itemize}
\item {Basic formalism}
\item {Representation of $\Psi_{sc}^+$ and $\Phi_0$}
\item {transforming from DVR to orbital basis}
\item{ extracting double ionization amplitudes}
\end{itemize}   
\end{abstract}
%
\maketitle
\tableofcontents
%
%
%
\section{Formulation of the problem for two active electrons in a single center DVR representation}
\label{sec:formulation}

We start with a reduction of the full double ionization problem to an effective two electron problem, treating two active electrons.  For example suppose we are treating one-photon double ionization to produce  the 1$^1$A$_1$ of H$_2$O$^++$.  If we freeze all the orbitals but the $1b_1$ orbital in the ground state configuration, $1a_1^2 \,  2a_1^2 \, 1b_2^2 \, 3a_1^2 \, 1b_1^2$ we would define the initial state including only correlation between the two $1b_1$ electrons as the singles and doubles CI wave function taking excitations only out of the two $1b_1$ spin orbitals.  If we express that wave function in terms of the natural orbitals of the singles and doubles CI, a theorem due to L\"owdin states that it has the form 
\begin{equation}
\begin{split}
\Phi_{gnd}   =& \sum_{n} C_{n}  |1a_1^2 \,  2a_1^2 \, 1b_2^2 \, 3a_1^2 \, n\alpha \, n\beta  |
\end{split}
\end{equation}
where there are only doubly occuppied natural orbitals in the list of determinants.  So we can pick the initial two electron state for our two-active-electrons description of the double ionization process as
\begin{equation}
\begin{split}
\Phi_0(1,2) &=  \sum_{n} C_{n}  |n\alpha(1)  \, n\beta(2)  | \\
& =  \sum_{n} C_{n} \phi_n(\mathbf{r}_1) \phi_n(\mathbf{r}_2) \frac{\alpha(1)\beta(2)-\beta(1)\alpha(2))}{\sqrt{2}}
\end{split}
\end{equation}
which we can represent on the DVR grid using the single center expansion of the natural orbitals
\begin{equation}
\phi_n(\mathbf{r}) = \sum_{l,m} R^{(n)}_{lm}(r) X_{lm} (  \mathbf{\hat{r}} )
\end{equation}
where $X_{lm}$ denotes a real spherical harmonic
\begin{equation}
\begin{split}
\Phi_0(1,2) &=  \sum_{n} C_{n}  |n\alpha(1)  \, n\beta(2)  | \\
& =  \sum_{n} C_{n} \phi_n(\mathbf{r}_1) \phi_n(\mathbf{r}_2) \\
& =  \sum_{\ell,m,\ell',m'} \left[  \sum_{n} C_{n} R^{(n)}_{\ell m}(r_1)  R^{(n)}_{\ell' m'}(r_2)\right]X_{\ell m} (  \mathbf{\hat{r}}_1 )X_{\ell' m'} (  \mathbf{\hat{r}}_2 ) \\
 \textrm{or} \\
\Phi_0(\mathbf{r}_1,\mathbf{r}_2) &= \sum_{\ell,m,\ell',m'} \EuScript{R}_{\ell m; \ell' m'}(r_1,r_2) \, X_{\ell m} (  \mathbf{\hat{r}}_1 )X_{\ell' m'} (  \mathbf{\hat{r}}_2 ) 
\end{split}
\end{equation}
where the radial functions are defined by the expression in square brackets, and clearly have the symmetries $\EuScript{R}_{\ell m ;\ell' m'}(r_1,r_2)=\EuScript{R}_{\ell m; \ell' m'}(r_2,r_1)$ and $\EuScript{R}_{\ell m; \ell' m'}(r_1,r_2)=\EuScript{R}_{\ell' m' ;\ell m}(r_1,r_2)$.
\\
\


\noindent \textbf{Code Check:}  in the DVR basis the two-electron radial functions are
\begin{equation}
\begin{split}
 \EuScript{R}_{\ell,m;\ell',m'} (r_1,r_2) = \sum_{i;j}  \EuScript{R}_{\ell,m;\ell',m'}^{i;j} \frac{\chi_i(r_1)}{r_1} \frac{ \chi_j(r_2) }{r_2}
\end{split}
\end{equation}
So the one-electron density matrix in this representation is
\begin{equation}
\begin{split}
\rho_{i,\ell,m;i',\ell',m'} = \sum_{j',\ell'',m''} \EuScript{R}_{\ell,m;\ell'',m''}^{i;j'}   \EuScript{R}_{\ell',m';\ell'',m''}^{i';j'} 
\end{split}
\end{equation}
So after constructing the representation of $\Phi_0$ from Gaussian orbitals, we can construct the density matrix and diagonalize it to get back the same naural orbital occupations as its eigenvalues that we started with from the Gaussian natural orbitals.  If there are N natural orbitals, this large matrix should have only N nonzero eigenvalues, to within the various numerical errors of the re-expansion of the Gaussian orbitals in the DVR basis.
\\
\

There is a possibility that I've still not worked through that we might be able to do this kind of reduction to two active electrons when ionizing one electron from each of two different orbitals.  The CI would look something like
\begin{equation}
\begin{split}
\Phi_{gnd}   =& \sum_{n} C_{n}  |1a_1^2 \,  2a_1^2 \, 1b_2^2 \,\, 3a_1\alpha  \, n\beta \,\, 1b_1 \alpha \, m\beta  |
\end{split}
\end{equation}
although there would be more than one determinant per configuration.  This would be a CI in which we might be able to use ``group restrictions'' in a CI code to restrict the active space to always have one electron in the singly target orbitals.  We would then find the natural orbitals and use them to define an effective two-electron initial state $\Phi_0$.  None of this has been worked out and it might not be possible.

With the right hand side constructed with this function we solve the driven Schr\"odinger equation for the scattered wave
\begin{equation}
(E_0+\hbar \omega -H_{eff})\Psi_{sc}^+(\mathbf{r}_1,\mathbf{r}_2) = \mu \Phi_0(\mathbf{r}_1,\mathbf{r}_2)
\label{eq:drivenSE}
\end{equation}
with, in this case,
\begin{equation}
H_{eff} = \hat{T}_1 + \hat{T}_2 + V_{nuc} + \sum_o \left( 2 \hat{J}_o - \hat{K}_o \right) + \frac{1}{r_{12}} 
\end{equation}
and then extract the double ionization amplitudes from $\Psi_{sc}^+(\mathbf{r}_1,\mathbf{r}_2) $, which may be problematic. 
We can also time propagate to obtain 
\begin{equation}
\Psi(\mathbf{r}_1,\mathbf{r}_2,t) = \exp[-iH_{eff})t]\,\mu \Phi_0(\mathbf{r}_1,\mathbf{r}_2)
\label{eq:prop}
\end{equation}
 and project on it with a product of Coulomb functions at long enough times to extract the double ionization amplitudes.                                                                                                                  

\section{Orbital basis in place of the DVR}
\label{sec:orbbasis}
We need to solve Eqs.(\ref{eq:drivenSE}) or (\ref{eq:prop})  in a space orthogonal to the doubly occupied orbitals that were frozen in this model.  We could do that by constructing the DVR representation of a projection on the doubly occupied orbitals
\begin{equation}
\begin{split}
\hat{P}_o(i) & \equiv  \langle \mathbf{r}_i | \phi_o \rangle \langle \phi_o|\mathbf{r'}_i \rangle \\
\hat{Q}_o &= 1 - \hat{P}_o(1) -  \hat{P}_o(2 ) + \hat{P}_o(1)\hat{P}_o(2)
\end{split}
\end{equation}
The $\hat{Q}_o$ operator is a two-electron projector that projects out orbital $\phi_o$. The full projector to project out a set of orbitals would then be
\begin{equation}
\begin{split}
\hat{Q}&= 1 - \sum_o (  \hat{P}_o(1) +  \hat{P}_o(2 ) - \hat{P}_o(1)\hat{P}_o(2))
\end{split}
\end{equation}
Which we would build in the DVR, again using $\phi_o(\mathbf{r}) = \sum_{l,m} R^{(o)}_{lm}(r) X_{lm} (  \mathbf{\hat{r}} )$.  To use it we would have to operate with $Q$ to construct $\hat{Q} H_{eff} \hat{Q}$ and operate in that rank defficient DVR representation. 

 This way of doing the projection is more cumbersome than just expanding $\Psi_{sc}^+(\mathbf{r}_1,\mathbf{r}_2) $ in a basis of orbitals that are orthogonal to the doubly occupied orbitals.  Whether the two active electrons are singlet or triplet coupled, the wave function can be expanded as
\begin{equation}
\Psi_{sc}^+(\mathbf{r}_1,\mathbf{r}_2) = \sum_{a,b} \phi_a(\mathbf{r}_1)  \phi_b(\mathbf{r}_2) \Psi_{a,b}
\end{equation}
The spin and space symmetry is carried in the coefficients $ \Psi_{a,b}$, and we would get them by projecting in the product orbital basis on Eq(\ref{eq:drivenSE}) for example
\begin{equation}
\langle \phi_c \phi_d | (E_0+\hbar \omega -H_{eff} )| \phi_a \phi_b \rangle \Psi_{a,b} = \langle \phi_c \phi_d | \mu \Phi_0 \rangle
\label{eq:orbdrivenSE}
\end{equation}
We would need for the orbital basis to be orthogonal, to make the one-body terms have the structure they have in the primitive DVR, but given our representation of $\Phi_0$ and $\mu$ in the DVR, this approach means just transforming the right hand side with the orbital coefficients.

 We also may want to project out not only the doubly occupied orbitals but also parts of the DVR space that corresponds to very high kinetic energy or one-body eigenvalues.   Suppose we construct a complete orbital basis and then represent Eq.(\ref{eq:drivenSE}) in that orbital space?  What would that entail?


\subsection{Nonspherical orbital basis: Method 1 requiring diagonalization of full one-body FEM-DVR Hamiltonian}
\label{sec:nonspherical}

One way to do this is to construct the eigenfunctions of the one-body Hamiltonian in the space orthogonal to the doubly occupied orbitals. 
The one-body operator is
\begin{equation}
h(\mathbf{r})= \hat{T}_1 + V_{nuc}(\mathbf{r})  + \sum_o \left( 2 \hat{J}_o(\mathbf{r}) - \hat{K}_o(\mathbf{r})  \right) 
\end{equation}
If there are M doubly occupied orbitals, $\phi_o(\mathbf{r})$,  to project them out of the space we need to represent them in the FEM-DVR basis with real spherical harmonics, $X_{\ell m}(\mathbf{\hat{r}})$, and then we would construct the projection operator
\begin{equation}
\begin{split}
\phi_o(\mathbf{r}) &= \sum_{\ell,m} R^{(o)}_{\ell m}(r) X_{\ell m} (  \mathbf{\hat{r}} )\\
\hat{Q}&= 1 - \sum_o \sum_{\ell,m} R^{(o)}_{\ell m}(r) X_{\ell m} (  \mathbf{\hat{r}} ) \sum_{\ell',m'} R^{(o)}_{\ell' m'}(r') X_{\ell' m'} (  \mathbf{\hat{r'}} ) \\
&= 1- \sum_{\ell,m,\ell',m'} \left[\sum_{o=1}^M  R^{(o)}_{\ell m}(r) R^{(o)}_{\ell' m'}(r') \right] X_{\ell m} (  \mathbf{\hat{r}} ) X_{\ell' m'} (  \mathbf{\hat{r'}} ) \\
&= 1- \sum_{\ell,m,\ell',m'}  \EuScript{R}_{\ell,m,\ell',m'} (r,r') X_{\ell m} (  \mathbf{\hat{r}} ) X_{\ell' m'} (  \mathbf{\hat{r'}} )
\end{split}
\label{eq:Qonebody}
\end{equation}
Where the radial functions are the part in square brackets in Eq.(\ref{eq:Qonebody}),
\begin{equation}
\EuScript{R}_{\ell,m,\ell',m'} (r,r') = \sum_{o=1}^M  R^{(o)}_{\ell m}(r) R^{(o)}_{\ell' m'}(r') 
\end{equation}
In the radial DVR basis, which we will assume here is the same for all $\ell, m$, the two-electron radial functions that define the projector have the representation
\begin{equation}
\begin{split}
 \EuScript{R}_{\ell,m,\ell',m'} (r,r') = \sum_{i,j}  \EuScript{R}_{\ell,m,\ell',m'}^{i,j} \frac{\chi_i(r)}{r} \frac{ \chi_j(r') }{r'}
\end{split}
\end{equation}

So in the single center DVR basis we are using, $\chi_i(r) X_{\ell,m} (\mathbf{\hat{r}})$ The Q operator is the matrix
\begin{equation}
\begin{split}
\hat{Q}_{i,\ell,m ;  j, \ell',m'} 
&= \delta_{\ell,\ell'} \delta_{m,m'} \delta_{i,j}  - \EuScript{R}_{\ell,m,\ell',m'}^{i,j} 
\end{split}
\end{equation}

We start with the one-body Hamiltonian in this primitive DVR basis, so the projected Hamiltonian is
\begin{equation}
\begin{split}
& \tilde{h}_{i,\ell,m, j, \ell',m'} = \\
& \sum_{\begin{matrix} k, \ell'',m'' \\ l,\ell''',m''' \end{matrix}} \hat{Q}_{i,\ell,m, k, \ell'',m''}\, h_{k,\ell'',m'', l, \ell''',m'''} \,\hat{Q}_{l,\ell''',m''', j, \ell',m'} 
 \end{split}
\end{equation}
and should have exactly $M$ zero eigenvalues, and the rest of the eigenvectors, 
\begin{equation}
\begin{split}
& \sum_{j,\ell',m'}\tilde{h}_{i,\ell,m, j, \ell',m'} R^{(a)}_{j,\ell',m'} = \epsilon_a \,R^{a}_{i,\ell,m} \\
&\phi_a(\mathbf{r}) = \sum_{i,\ell,m} \,R^{(a)}_{i,\ell.m}\,  \frac{\chi_i(r)}{r} X_{\ell,m}
 \end{split}
\end{equation}
would be the complete projected orbital basis.  [Note we use $i j k l$ to denote DVR function indices and $\ell, m$ and primed versions of those indices to denote angular momenta.]
\\
\

\noindent \textbf{Code check:} There are 4 doubly occupied orbitals in the water case, but to check project out just 1, 2, 3 and 4 of them to check that the number of zero eigenvalues is the number of orbitals projected out.  This check could be done without finding the eigenvectors.
\\
\

This is a completely equivalent basis to our original DVR basis, and the orthonormal orbitals $\phi_\alpha(\mathbf{r})$ are nonspherical, but they also don't have any particular physical meaning (except for the occupied orbitals that we projected out).  

Moreover the one-body Hamiltonian is a fairly large matrix with coupled $\ell m$ blocks. Suppose we need to use up to $\ell = 10$, so that there would be $11^2 = 121$ $\ell m$ pairs.  If there are 500 radial DVR points, then the matrix $h$ is 60,500 $\times$ 60,500, and we would need all of the eigenvectors.  That might be possible, but it doesn't seem optimum.   There are at least one other way to proceed to generate an orbital basis that is orthogonal to the doubly occupied orbitals.



\subsection{Nonspherical orbital basis: Method 2 using eigenvectors of each $\ell,m$ block of the one-body Hamiltonian}
\label{sec:lorbs}


Our projection operator to project out the doubly occupied orbitals from a one-electron function in Eq.(\ref{eq:Qonebody}) has the form
\begin{equation}
\begin{split}
\hat{Q}&=  \sum_{i ,\ell,m, j, \ell',m'} \frac{\chi_i(r)}{r} X_{\ell m} (  \mathbf{\hat{r}} )\hat{Q}_{i,\ell,m ; j, \ell',m'}   \frac{ \chi_j(r') }{r'} X_{\ell' m'} (  \mathbf{\hat{r'}} ) \\
\end{split}
\label{eq:Q1}
\end{equation}
It is notably \textbf{not} diagonal in angular momenta.  If we were to orthogonalize the basis in each angularm momentum block $\ell,m$ to the $\ell,m$ component of an orbital $\phi_a(\mathbf{r})$, the resulting basis would be orthogonal to $\phi_a(\mathbf{r})$ but also to other parts of the space not orthogonal to that orbital.  So we cannot project out each diagonal angular momentum component of $\hat{Q}$ and contruct the orthogonalized basis one partial wave at a time.  However we can contruct a basis that can be used to filter out high lying one-body energy eigenvectors, and is also orthogonal to the doubly occupied orbitals,  by starting with diagonalizing the diagonal blocks of the one-body Hamiltonian.

\begin{enumerate}
\item{ Diagonalize each diagonal block with respect to $\ell, m$ of the one-body Hamiltonian in the primitive DVR basis
\begin{equation}
\begin{split}
 h_{i,\ell,m, j, \ell',m'} &\equiv \int \frac{\chi_i(r)}{r} X_{\ell m}(\mathbf{\hat{r}}) \, \hat{h}(\mathbf{r}) \,  \frac{\chi_j(r)}{r} X_{\ell' m'}(\mathbf{\hat{r}}) d^3r\\
& \sum_{j}h_{i,\ell,m, j, \ell,m} R^{(n)}_{j,\ell m} = \epsilon_{n}^{\ell m} \, R^{(n)}_{i,\ell, m} \\
&\mathbf{R}_{\ell m} = \left(  \vec{R}^{(1)}_{\ell m},  \vec{R}^{(2)}_{\ell m},  \vec{R}^{(3)}_{\ell m}, \cdots, \vec{R}^{(N_{DVR})}_{\ell m} \right)
 \end{split}
\end{equation}
Each $\ell,m$ block of $h$ is of dimension $N_{DVR} \times N_{DVR}$, so for $\ell = 10$ with $N_{DVR} = 500$, this would be 121 diagonalizations of $500 \times 500$ matrices with computation of the eigenvectors.  That's easy.

Moreover the high lying eigenvalues of the diagonal blocks will be close to the high lying eigenvalues of the full one-body Hamiltonian, so they are likely to be a good proxy for the high frequency components of the full problem.
}
\item{Arrange the eigenvectors of the individual blocks to make a complete orbital space basis
\begin{equation}
\boldsymbol{\mathcal{R}}=\begin{pmatrix}
\mathbf{R}_{0 0} & 0&0&0& \cdots \\
0&\mathbf{R}_{1 -1} &0&0&\cdots \\
0&0&\mathbf{R}_{1 0} &0&\cdots \\
0&0 &0&\mathbf{R}_{1 0}&\cdots \\
\vdots&\vdots&\vdots&\vdots&\ddots
\end{pmatrix}
\end{equation}
The dimension of this matrix is $N_{DVR} N_{\ell m} \times N_{DVR} N_{\ell m}$ if all the eigenvectors are kept. If we want to drop eigenvectors 
with high values of $\epsilon_{n}^{\ell m}$, we drop the corresponding columns of $\boldsymbol{\mathcal{R}}$.  The matrix is indexed as
$\mathcal{R}_{i \ell m}^{a}$, with $a = 1,2 \cdots \sum_{\ell m}M_{\ell m}$, where $M_{\ell m}$ is the number of eigenvectors kept in  each $\ell,m$ block.
}
\item{Apply the projection operator to find the coefficients of the orthogonal orbital basis,
\begin{equation}
\tilde{\mathcal{R}}_{i,\ell,m}^a=\sum_{j,\ell',m'} \hat{Q}_{i,\ell,m ; j, \ell',m'} \mathcal{R}_{j,\ell',m'}^{a}
\end{equation}
This is the projected basis. Note that for any given value of $a$, $ \mathcal{R}_{j,\ell',m'}^{a}$ has nonzero values for only one value of $\ell' m'$, where as the projected functions given by $\tilde{\mathcal{R}}_{i,\ell,m}^a$ have nonzero values in all $\ell m$ blocks.
}
\item{Schmidt orthogonalize:  Because this basis does not come from diagonalizing an operator it is not necessarily orthogonal.  While It may be nearly orthogonal, one last step is necessary to make that orthonormality exact.  So
\begin{equation}
\bar{\bar{{\mathcal{R}}}}_{i,\ell,m}^a = \textrm{Gram-Schmidt orthogonalize}\quad \tilde{\mathcal{R}}_{i,\ell,m}^a 
\end{equation}
}
\end{enumerate}
This basis does \textbf{not} diagonalize the diagonal blocks of the one-body Hamiltonian any longer.  Any way of building an orthonormal orbital basis, other than diagonalizing the full, projected one-body Hamiltonian matrix and finding all the eigenvectors, will leave coupling in the one-body Hamiltonian matrix.  The original one-body Hamiltonian $h_{i \ell m;j \ell' m'}$ in the FEM-DVR basis also has couplings between the $\ell m$ blocks because of the nonspherical nuclear potential and Coulomb and exchange operators from the frozen orbitals.  So this transformation doesn't introduce any new complications, except the fact that $\tilde{h}$ is rank deficient.


\section{Hamiltonian matrix elements and their transformation to the orbital basis}


\subsection{Kinetic energy:}  
The real spherical harmonics are eigenfunctions of $\hat{L}^2$ so the kinetic energy matrix
elements don't change from their form for complex spherical harmonics.
\begin{equation}
T_{i,\ell,m;j,\ell',m'}^{(\ell)} =  T_{i,j}^{(\ell)}  \delta_{\ell,\ell'} \delta_{m,m'}
\end{equation}
with
\begin{equation}
T_{i,j}^{(\ell)} = -\frac{1}{2} \int_0^{r_{max}} \chi_i(r)
\left(\frac{d^2}{dr^2}-\frac{\ell(\ell+1)}{r^2}\right)
\chi_j(r) dr
\end{equation}
To transform the kinetic energy alone to the orbital basis if we need it
\begin{equation}
\begin{split}
T_{a,a'}&=  \sum_{i,\ell,m,j,\ell',m'} \bar{\bar{{\mathcal{R}}}}_{i,\ell,m}^a T_{i,j}^{(\ell)}  \delta_{\ell,\ell'} \delta_{m,m'}\bar{\bar{{\mathcal{R}}}}_{j,\ell',m'}^{a'}\\
&=  \sum_{i,\ell,m,j} \bar{\bar{{\mathcal{R}}}}_{i,\ell,m}^a T_{i,j}^{(\ell)} \bar{\bar{{\mathcal{R}}}}_{j,\ell,m}^{a'} \\
&= \sum_{i,\ell,j}  A^{a,a'}_{i,j,\ell}T_{i,j}^{(\ell)} \qquad  A^{a,a'}_{i,j,\ell} \equiv\sum_{m=-\ell}^\ell  \bar{\bar{{\mathcal{R}}}}_{i,\ell,m}^a  \bar{\bar{{\mathcal{R}}}}_{j,\ell,m}^{a'}
\end{split}
\end{equation}
requires two steps that scale like $N^2 \times N_{DVR}^2 \times N_\ell$, where $N$ is the total dimension of the FEM-DVR basis $N = N_{DVR} \times N_{\ell m}$.  The kinetic energy matrix $T_{a,a'}$ is now a full matrix, not block diagonal by $\ell$. 
 

\subsection{Nuclear attraction potential}
The nuclear attraction potential is
\begin{equation}
V_{nuc}(\mathbf{r}) = \sum_\beta \frac{Z_\beta}{|\mathbf{r}-\mathbf{R}_\beta|}
\end{equation}
To perform the integrals 
\begin{equation}
\begin{split}
& V^{nuc}_{i, \, \ell, \,m; \, j, \,\ell', \, m'}(\mathbf{R}_\beta) = \\
&\int \frac{\chi_{i}(r)}{r}X_{\ell, \,m} (\mathbf{\hat{r}})
\frac{1}{|\mathbf{r}-\mathbf{R}_\beta|}
\frac{\chi_j(r)}{r} X_{\ell', \, m'} (\mathbf{\hat{r}})
d\mathbf{r}
\end{split}
\end{equation}
we need the multipole expansion in terms of the real spherical harmonics, which is
\begin{equation}
\frac{1}{| \mathbf{r}- \mathbf{r}'|} = 
 \sum_{\lambda,\mu} \frac{4 \pi}{2\lambda +1}
X_{\lambda,\mu}(\mathbf{\hat{r}})
X_{\lambda,\mu}(\mathbf{\hat{r}}')
\frac{r^\lambda}{r^{\lambda+1}}
\end{equation}
as shown in Appendix \ref{sec:angmom}.

We also need the integral of three real spherical harmonics
that is analogy of what is called a Gaunt coefficient for complex
spherical harmonics.  We will use the notation
$C(\ell m;\lambda \mu;\ell' m')$ for these  Gaunt coefficients 
here and in the two-electron integrals.
\begin{equation}
\begin{split}
& C(\ell m;\lambda \mu;\ell' m')= \\
& \int 
X_{\ell,m} (\mathbf{\hat{r}})
X_{\lambda,\mu} (\mathbf{\hat{r}})
X_{\ell',m'} (\mathbf{\hat{r}})
d\mathbf{\hat{r}} 
\end{split}
\end{equation}
Unlike the usual Gaunt coefficients, for which two of the spherical
harmonics are complex conjugated in the integral, these are symmetric
with respect to any permutation of pair of angular momentum indices.
With these we can assemble the basic integral
\begin{equation}
\boxed{
\begin{aligned}
& V^{nuc}_{i, \, \ell, \,m; \, j, \,\ell', \, m'}(\mathbf{R}_\beta) = \\
& \sum_{\lambda=|\ell-\ell'|}^{\ell+\ell'}
  U^\lambda_{i,j}(R_\beta) \\
& \times \left[
 \frac{4\pi} {2\lambda+1} 
\sum_{\mu=-\lambda}^{\lambda} X_{\lambda,\mu}(\mathbf{\hat{R}}_\beta)
C(\ell m;\ell' m';\lambda \mu)   \right] \\
\end{aligned}
}
\end{equation}
Note that the Gaunt coefficients are real.

That leaves us with the radial integral 
\begin{equation}
U^\lambda_{i,j}(R_\beta) = \int_0^{r_{max}} \chi_{i}(r) \chi_{j}(r) 
\frac{r_<^\lambda}{r_>^{\lambda+1}} 
\end{equation}
where the $r_<$ and $r_>$  refer to the less or greater of $r$ and $R_\beta$.
Using the Poisson equation approach, Eq.(66b) in the review \cite{TopicalReview2004}
(where there is a typo, the sum should be over $m$ not over $i$)
we get an explicit expression for this radial integral
\begin{equation}
\boxed{
\begin{aligned}
U^\lambda_{i,j}(R_\beta) & =  \\
\delta_{i,j}
& \left[
(2 \lambda +1) \sum_{m=1}^N 
\frac{\chi_m(R_\beta)}{R_\beta}
\left[ T^{(\lambda)}_{m,j}\right]^{-1} 
\frac{1}{r_{j}\sqrt{\rm{w}_{j}}} 
 \right. \\
& \left.  
 + \frac{R_\beta^{\lambda} r_{j}^\lambda }{r_{max}^{2\lambda+1}} 
\right]
\end{aligned}
}
\end{equation}
Note that the nuclear attraction integrals are diagonal in the 
DVR radial functions, but they are not diagonal in the angular momentum
indices.  

We would first sum over the nuclei, of course
\begin{equation}
 V^{nuc}_{i, \, \ell, \,m; \, j, \,\ell', \, m'} = \sum_\beta V^{nuc}_{i, \, \ell, \,m; \, j, \,\ell', \, m'}(\mathbf{R}_\beta) 
\end{equation}
which would scale like $N^2$
Then the transformation the nuclear attraction potential (or the full one-body Hamiltonian) to the orbital basis
has the form
\begin{equation}
\begin{split}
V^{nuc}_{a,a'} &=  \sum_{i,\ell,m,j,\ell',m'} \bar{\bar{{\mathcal{R}}}}_{i,\ell,m}^a  V^{nuc}_{i, \, \ell, \,m; \, j, \,\ell', \, m'} \bar{\bar{{\mathcal{R}}}}_{j,\ell',m'}^{a'}\\
\end{split}
\end{equation}
which we would organize into two steps
\begin{equation}
\begin{split}
V^{temp}_{a',i, \, \ell, \,m}&=  \sum_{j,\ell',m'} V^{nuc}_{i, \, \ell, \,m; \, j, \,\ell', \, m'}  \bar{\bar{{\mathcal{R}}}}_{j,\ell',m'}^{a'}\\
V^{nuc}_{a,a'} &=  \sum_{a',i,\ell m} \bar{\bar{{\mathcal{R}}}}_{i,\ell,m}^a   V^{temp}_{a',i, \, \ell, \,m}\
\end{split}
\end{equation}
each scaling as $N^3$.

With these formulas we can construct the matrix of the one-electron Hamiltonian,$h_{i \ell m, j \ell' m'}$, and transform it to the orbital basis, except for the Coulomb and exchange operators.   Before turning to them, let's look at the two electron integrals, since we are going to need them for the Coulomb and exchange interactions with the core.

\subsection{Two-electron integrals}

The generic two-electron integral in terms of radial DVR 
basis functions and spherical harmonics is familiar from our earlier work on atoms and diatomics, but there we 
used complex spherical harmonics.  In terms of real spherical harmonics we now compute the integrals (in $\langle 1 1|| 22 \rangle$ notation 
with no complex conjugates, because we are using real spherical harmonics)
\begin{equation}
\begin{split}
&\left< 
i ,\, \ell_1, \, m_1, \, k ,\, \ell'_1, \, m'_1, \, ||
j, \,\ell_2, \, m_2, \,  l, \,\ell'_2, \, m'_2 \right>
 = \\
&\int 
\phi_{i}(r_1)X_{\ell_1,m_1} (\mathbf{\hat{r_1}})
\phi_{k}(r_1)X_{ \ell'_1, \, m'_1} (\mathbf{\hat{r_1}})
\\
& \times
\frac{1}{|\mathbf{r}_1-\mathbf{r}_2|}
 \phi_{j}(r_2) X_{\ell_2,m_2} (\mathbf{\hat{r_2}}) 
\phi_{l}(r_2) X_{\ell'_2, \, m'_2 } (\mathbf{\hat{r_2}})
d\mathbf{r_1}
d\mathbf{r_2}
\end{split}
\end{equation}
where the volume elements are $dr \sin \theta d\theta d\phi$ because the radial functions were $\phi_i(r)/r$.
This is the matrix element of $1/r_{12}$ between configurations $|i ,\, \ell_1, \, m_1;j, \,\ell_2, \, m_2,|$ and $|  k ,\, \ell'_1, \, m'_1; l, \,\ell'_2, \, m'_2 |$.
Using the multipole expansion and performing the angular 
integrations we get an expression in terms of Gaunt coefficients for real spherical harmonics.
\begin{equation}
\begin{split}
&\left< 
i ,\, \ell_1, \, m_1, \, k ,\, \ell'_1, \, m'_1, \, ||
j, \,\ell_2, \, m_2, \,  l, \,\ell'_2, \, m'_2 \right>
 = \\
& \sum_{\lambda=|\ell_1-\ell_1'|}^{|\ell_1-\ell_1'|}
 \left< \phi_{i} \phi_{k}|\frac{r_<^\lambda}{r_>^{\lambda+1}}| 
\phi_{j}\phi_{l}\right> \\
& \times \frac{4\pi} {2\lambda+1} 
\sum_{\mu=-\lambda}^{\lambda} 
 C(\ell_1 m_1;\lambda \mu ;\ell_1' m_1') 
 C(\ell_2 m_2;\lambda \mu ;\ell_2' m_2') 
\end{split}
\end{equation}
Note that the Gaunt coefficients are zero if \textit{either} of the triples, $\ell_1,\lambda,\ell_1'$ or $\ell_2,\lambda,\ell_2'$ violate the triangle inequality, so the sum over $\lambda$ can be written as either of those ranges, under the assumption that the Gaunt coefficients will restrict the sum to the overlap of the two triangle inequality ranges for $\lambda$.


The radial integrals we need are given by the formulas in the 
the Topical Review \cite{TopicalReview2004} that come from 
the Poisson equation approach.  They are diagonal in the radial 
DVR indices, and are given by 
\begin{equation}
\begin{split}
& \left< \phi_{i} \phi_{k}|\frac{r_<^\lambda}{r_>^{\lambda+1}}| 
\phi_{j}\phi_{n_2}\right> \\
&= 
\delta_{i,k}\delta_{j,l}
\left(
\frac{(2\lambda+1)} { r_{k}\sqrt{w_{k}} r_{l} \sqrt{w_{l}}}
\left[T_{k,l}^{(\lambda)}\right]^{-1}
+
\frac{ r_{k}^\lambda  r_{l}^\lambda}{r_{max}^{2\lambda+1}}
\right) \\
&= \delta_{i,k} \delta_{j,l} M^{(\lambda)}_{k,l}
\end{split}
\label{final2e}
\end{equation}
The matrices $M^{(\lambda)}_{k,l}$ are only $N_{DVR} \times N_{DVR}$ and are computed once and stored.  

\noindent \textbf{Transformation:}  To use the global orbital basis we need to transform the two electron integrals.  The general form of a 
four index transformation is
\begin{equation}
\begin{split}
\langle a_1 a_2 || a_3 a_4 \rangle &= \\
\sum_{i,\ell_1,m_1} &\sum_{k,\ell'_1,m'_1}\sum_{j,\ell_2,m_2} \sum_{l,\ell'_2,m'_2}
\bar{\bar{{\mathcal{R}}}}_{i,\ell_1,m_1}^{a_1} 
\bar{\bar{{\mathcal{R}}}}_{k,\ell'_1,m'_1}^{a_2}  \\
&\left<  i ,\, \ell_1, \, m_1, \, k ,\, \ell'_1, \, m'_1, \, ||
j, \,\ell_2, \, m_2, \,  l, \,\ell'_2, \, m'_2 \right> \\
&\times \bar{\bar{{\mathcal{R}}}}_{j,\ell_2,m_2}^{a_3} 
\bar{\bar{{\mathcal{R}}}}_{l,\ell'_2,m'_2}^{a_4} 
\end{split}
\end{equation}
and with using the diagonal property in $i,k$ and $j,l$ we would do this in four steps the first of which would be 
\begin{equation}
\begin{split}
&\left< a_1, \, k ,\, \ell'_1, \, m'_1, \,  || j, \,\ell_2, \, m_2, \,  l, \,\ell'_2, \, m'_2 \right> = \\
&\sum_{i,\ell_1,m_1} 
\bar{\bar{{\mathcal{R}}}}_{i,\ell_1,m_1}^{a_1} 
\\
&\left<  i ,\, \ell_1, \, m_1, \, k ,\, \ell'_1, \, m'_1, \, ||
j, \,\ell_2, \, m_2, \,  l, \,\ell'_2, \, m'_2 \right> \\
\end{split}
\end{equation}
\textcolor{blue}{Each step scales like $N^5$, where $N$ is the total size of the FEM-DVR and angular product basis.  In principle it requires storage for
two sets of two-electron integrals of $N^4$ and one matrix of orbital coefficients of $N^2$.  In our estimate of 500 radial points and 10 angular momenta that is $60,500^4 = 1/3\times 10^{19}$, so if we were to do this without taking advantage of the structure of the  primitive DVR two-electron integrals we would have to use one of the standard out-of-core four-index transformation algorithms that are used in quantum chemistry codes.}

It may be worthwhile asking what would happen if instead of a basis of nonspherical orbitals we had a separate radial basis for each $\ell,m$ pair.  That's the same number of orbitals of course, but then the transformation would be done for every set of four angular momentum indices separately
\begin{equation}
\begin{split}
&\langle a_1,\ell_1,m_1; a_2 ,\ell'_1,m'_1|| a_3 \ell_2,m_2; a_4\ell'_2,m'_2 \rangle = \\
&\sum_{i,j,k,l}
\bar{\bar{{\mathcal{R}}}}_{i,\ell_1,m_1}^{a_1} 
\bar{\bar{{\mathcal{R}}}}_{k,\ell'_1,m'_1}^{a_2}  \\
&\left<  i ,\, \ell_1, \, m_1, \, k ,\, \ell'_1, \, m'_1, \, ||
j, \,\ell_2, \, m_2, \,  l, \,\ell'_2, \, m'_2 \right> \\
&\times \bar{\bar{{\mathcal{R}}}}_{j,\ell_2,m_2}^{a_3} 
\bar{\bar{{\mathcal{R}}}}_{l,\ell'_2,m'_2}^{a_4} 
\end{split}
\end{equation}
\textcolor{blue}{\textbf{\textit{and}} the DVR integrals are diagonal in $i,k$ and $j,l$ so each transformation scales as $N_{DVR}^5$.  We could build each block of the two-electron Hamiltonian separately, calculating the DVR integrals we need on the fly.}

\textcolor{red}{So the orbital basis idea becomes completely practical if we have an orbital basis for each $\ell,m$ pair.  We could do that easily if we weren't projecting out nonspherical doubly occupied orbitals.}


\subsubsection{$\left( 2 \hat{J}_o - \hat{K}_o \right)$} 



% appendices may preceed or follow the references
\appendix
\section{Identities involving angular momentum}
\label{sec:angmom}
Definition of real spherical harmonics in terms of the complex spherical harmonics with the Condon phases,
\begin{equation}
\begin{split}
&Y_{l,m}(\theta,\phi) =\\
&\begin{cases} (-1)^m  \left[ \frac{2l+1}{2} \frac{(l-|m|)!}{(l-|m|)!} \right]^{1/2} P_l^{|m|}(\cos \theta) \frac{1}{\sqrt{2\pi}} e^{i\, m \, \phi} & m>0 \\ 
\qquad \quad   \left[ \frac{2l+1}{4 \pi}  \right]^{1/2} P_l^{|m|}(\cos \theta)  & m=0 \\
\qquad \quad   \left[ \frac{2l+1}{2} \frac{(l-|m|)!}{(l-|m|)!} \right]^{1/2} P_l^{|m|}(\cos \theta)   \frac{1}{\sqrt{2\pi}} e^{i\, m \, \phi} & m < 0
\end{cases}
\end{split}
\end{equation}
is
\begin{equation}
\begin{split}
&X_{l,0}(\theta,\phi) = Y_{l,0}(\theta,\phi) \\
&X_{l,+|m|}(\theta,\phi) = \frac{1}{\sqrt{2}} \left((-1)^mY_{l,|m|}(\theta,\phi) + Y_{l,-|m|}(\theta,\phi) \right)\\
&X_{l,-|m|}(\theta,\phi) = \frac{1}{i\sqrt{2}} \left((-1)^mY_{l,|m|}(\theta,\phi) -Y_{l,-|m|}(\theta,\phi) \right)\\
& Y_{l,|m|}(\theta,\phi) = \frac{1}{\sqrt{2}} (-1)^m (X_{l,+|m|}(\theta,\phi)+i X_{l,-|m|}(\theta,\phi)) \\
& Y_{l,-|m|}(\theta,\phi) = \frac{1}{\sqrt{2}}  (X_{l,+|m|}(\theta,\phi)-iX_{l,-|m|}(\theta,\phi)) \\
\end{split}
\end{equation}
This definition makes the $+|m|$ real spherical harmonic the cosine-like one and the $-|m|$ real spherical harmonic the sine-like one.

The multipole expansion of $1/r_{12}$ is then the same in terms of complex or spherical harmonics 
\begin{equation}
\begin{split}
\frac{1}{| \mathbf{r}- \mathbf{r}'|} &= 
 \sum_{\lambda,\mu} \frac{4 \pi}{2\lambda +1}
Y_{\lambda,\mu}(\mathbf{\hat{r}})^*
Y_{\lambda,\mu}(\mathbf{\hat{r}}')
\frac{r^\lambda}{r^{\lambda+1}} \\
&= 
 \sum_{\lambda,\mu} \frac{4 \pi}{2\lambda +1}
X_{\lambda,\mu}(\mathbf{\hat{r}})
X_{\lambda,\mu}(\mathbf{\hat{r}}')
\frac{r^\lambda}{r^{\lambda+1}} 
\end{split}
\end{equation}
as can be seen by substitution.

Using
\begin{equation}
Y^*_{\ell,m} = (-1)^m Y_{\ell,-m} 
\end{equation}
The real spherical harmonics can also be written
\begin{equation}
\begin{split}
&X_{l,0}(\theta,\phi) = Y_{l,0}(\theta,\phi) \\
&X_{l,+|m|}(\theta,\phi) = \frac{(-1)^m}{\sqrt{2}} \left(Y_{l,|m|}(\theta,\phi) + Y^*_{l,|m|}(\theta,\phi) \right)\\
&X_{l,-|m|}(\theta,\phi) = \frac{(-1)^m}{i\sqrt{2}} \left(Y_{l,|m|}(\theta,\phi) -Y^*_{l,|m|}(\theta,\phi) \right)\\
& Y_{l,|m|}(\theta,\phi) = \frac{(-1)^m }{\sqrt{2}} (X_{l,+|m|}(\theta,\phi)+i X_{l,-|m|}(\theta,\phi)) \\
& Y^*_{l,|m|}(\theta,\phi) = \frac{(-1)^m}{\sqrt{2}}  (X_{l,+|m|}(\theta,\phi)-iX_{l,-|m|}(\theta,\phi)) \\
\end{split}
\end{equation}

We will use the notation
$C(j_1\mu_1|j_1'\mu_1',\lambda \mu)$ for the Gaunt coefficients 
in terms of real spherical harmonics
\begin{equation}
\begin{split}
& C(j_1\mu_1|j_1'\mu_1',\lambda \mu) = \\
& \int 
X_{\lambda,\mu} (\mathbf{\hat{r}})
X_{j_1',\mu_1'} (\mathbf{\hat{r}}) X_{j_1,\mu_1} (\mathbf{\hat{r}})
d\mathbf{\hat{r}}  
\end{split}
\end{equation}
These integrals can be related to the basic integral of three complex spherical
harmonics, but I haven't found a simpler expression than one involving all eight terms
for each case of positive or negative m values.
\begin{equation}
\begin{split}
 \int 
Y_{\lambda,\mu} (\mathbf{\hat{r}})&
Y_{j_1',\mu_1'} (\mathbf{\hat{r}})
Y_{j_1,\mu_1} (\mathbf{\hat{r}})
d\mathbf{\hat{r}}  = \\
&
\left[\frac{(2j_1' + 1)(2j_1+1)(2\lambda+1)}{4\pi}\right]^{1/2} \\
& \times 
\begin{pmatrix}
\lambda&j_1'&j_1\\
0&0&0
\end{pmatrix}
\begin{pmatrix}
\lambda&j_1'&j_1\\
-\mu&-\mu_1'&\mu_1
\end{pmatrix}
\end{split}
\end{equation}


\section{Other identies}
% make bibliography
\bibliography{DVR_doublePI.bib}
%
\end{document}  % 


